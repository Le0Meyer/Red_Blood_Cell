% Tout ce qui est mis derrière un « % » n'est pas vu par LaTeX
% On appelle cela des « commentaires ».  Les commentaires permettent de
% commenter son document - comme ce que je suis en train de faire
% actuellement - et de cacher du code - cf. la ligne \pagestyle.

\documentclass[a4paper,fleqn]{article}

% Options possibles : 10pt, 11pt, 11pt (taille de la fonte)
%                     oneside, twoside (recto simple, recto-verso)
%                     draft, final (stade de développement)

\usepackage[utf8]{inputenc}   % LaTeX, comprends les accents !
\usepackage[T1]{fontenc}      % Police contenant les caractères français
\usepackage[english,french]{babel}  % Placez ici une liste de langues, la dernière étant la langue principale
\usepackage{tabularx}

\usepackage[a4paper]{geometry}% Réduire les marges
%\pagestyle{headings}        % Pour mettre des entêtes avec les titres des sections en haut de page

\usepackage{amsmath} % package pour écrire des maths


\title{Modèle du globule rouge}           % Les paramètres du titre : titre, auteur, date
\author{Meyer Léo \and Klay Léna}
\date{}                       % La date n'est pas requise (la date du jour de compilation est utilisée en son absence)

\sloppy                       % Ne pas faire déborder les lignes dans la marge




\begin{document}

\maketitle                    % Faire un titre utilisant les données de  \title, \author et \date


\subsection*{Introduction}

Dans ce rapport, nous nous intéresserons aux variations des quantités physico-chimiques  du globule rouge humain, en nous appuyant sur le modèle écrit par Virgilio L.LEW et Robert M.BOOKCHIN. Le but d'un tel modèle est de quantifier et qualifier les flux d'espèces chimiques, les variations de volume et de pH, dans le globule rouge et son milieu environnant. Pour cela nous allons implémenter un système d'équations différentielles sous python et utiliser un solveur pour le résoudre.


\tableofcontents              % Table des matières

% \part{Titre}                % Commencer une partie...

\section{Modèle et notations}               % Commencer une section, etc.

Afin de caractériser les flux à travers la membrane plasmique du globule rouge, nous considérerons les espèces chimiques suivantes :

\begin{tabular}{p{11cm}cr}
\textbf{Espèces chimiques}                                                   & \textbf{Abréviation} \\
\\
Sodium                                                            & ${Na}$   \\
Potassium                                                         & $K$    \\
Magnesium                                                         & ${Mg}$   \\
Hydrogen                                                          & $H$    \\
Hémoglobine                                                       & ${Hb}$   \\
Anion perméant                                                    & $A$    \\
Anion imperméant intracellulaire considéré comme non protonisable & $X$    \\
Anion monovalent imperméant extracellulaire                       & $Y$    \\
Tampon imperméant extracellulaire                                 & $B$    \\
Forme protonisée du tampon                                        & ${HB}$   \\

\end{tabular}\\
\\


Nous définissons ensuite les variables d'intérêt du système, où $i$ correspond à une des espèces citées plus haut, dans la mesure où la variable s'applique à celle-ci: \\
\\

\begin{tabular}{p{8cm}lr}

\textbf{Variable}                                                      & \textbf{Abréviation}     & \textbf{Unité}      \\
\\
Quantité de l'espèce $i$ dans un litre de solution de cellules compactées  & $Q_i$                    & $mmol.loc^{-1}$     \\
Charge net de l'ion imperméant intracellulaire                         & $Q_{(-)}$               & $mEq.loc^{-1}$      \\
Concentration de l'espèce $i$ dans le cytoplasme                           & $C_{i}^{c}$              & $mmol.l^{-1}$       \\
Concentration de l'espèce $i$ dans le milieu extracellulaire               & $C_{i}^{m}$              & $mmol.l^{-1}$       \\
pH intracellulaire                                                     & $pH^c$                                         \\
pH extracellulaire                                                     & $pH^m$                                         \\
Volume du cytoplasme dans un litre de solution de cellules compactées & $V_w$                    & $l.loc^{-1}$        \\
Flux total de l'espèce $i$ (nécessairement perméant)                       & $\Phi_i$                 & $mmol.loc^{-1}.h^{-1}$\\
Flux partiel à travers la pompe Na: K de l'espèce $i$                      & $\Phi_{i}^{P}$           & $mmol.loc^{-1}.h^{-1}$\\
Flux partiel lié à la diffusion de l'espèce $i$                             & $\Phi_{i}^{L}$           & $mmol.loc^{-1}.h^{-1}$\\
Flux partiel lié à l'électro-diffusion de l'espèce $i$                      & $\Phi_{i}^{G}$           & $mmol.loc^{-1}.h^{-1}$\\
Flux partiel à travers le co-transporteur Na: K: 2A de l'espèce $i$         & $\Phi_{i}^{Co}$          & $mmol.loc^{-1}.h^{-1}$\\
Flux partiel à travers le co-transporteur H: A de l'espèce $i$              & $\Phi_{i}^{HA}$          & $mmol.loc^{-1}.h^{-1}$\\
Coefficient osmotique de l'hémoglobine                                 & $f_{Hb}$                                       \\

\end{tabular}



\begin{tabular}{p{8cm}cr}

\textbf{Paramètres}                                             & \textbf{Abréviation}     & \textbf{Unité}      \\
Taux d'hématocrite                                             & $Ht$                                           \\
Volume intracellulaire initial                                 & $V_{w}^{\left(0\right)}$                       \\
Perméabilité de l'espèce $i$ lors de la diffusion                  & $P_{i}^{L}$              & $h^{-1}$            \\
Perméabilité de l'espèce $i$ lors de l'électro-diffusion            & $P_{i}^{G}$              & $h^{-1}$            \\
Constante de turnover du co-transport Na: K: 2A (Co)            & $k_{Co}$                                       \\
Constante de turnover du co-transport H: A (HA)                 & $k_{HA}$                                       \\
Valence de l'ion $i$                                             & $z_i$                                          \\
Constante de Faraday                                           & $F$                      & $s.A.mol^{-1}$      \\
Constante des gaz parfaits                                     & $R$                      & $J.mol^{-1}.K^{-1}$ \\
Température absolue                                            & $T$                      & $K$                 \\
pH isoélectrique de l'hémoglobine                              & $pI$                                           \\
Constante de dissociation de la solution tampon B              & $K_B$                    & $mol.l^{-1}$        \\
Charge négative nette de l'hémoglobine lorsque $pH^c = pI + 1$ & $a$  & $Eq.mol^{-1}$  \\
Coefficients viriaux de l'équation de $f_{Hb}$                 & $b, c$                   &                     \\
Facteur de compensation dans le co-transport Na: K: 2A          & $d$                      &                     \\
Flux saturé de Na à travers la pompe Na: K                   & $\Phi_{max}^{Na}$        & $mmol.loc^{-1}.h^{-1}$ \\

\end{tabular}



\section{Équations constitutives}

\subsection{Quelques équations utiles pour la suite}

\begin{equation}
E =  \frac{RT}{F}\ln{\left(\frac{\dfrac{P_{Na}^G\,Q_N}{V_w} + \dfrac{P_K^G\,Q_K}{V_w} + P_A^G\,C_A^m}{P_{Na}^G\,C_{Na}^m + P_K^G\,C_K^m + \dfrac{P_A^G\,Q_A}{V_w}}\right)}
\end{equation}

\begin{equation}
{C_{H}^{c,m}=\exp_{10}{\left(-pH^{c,m}\right)}}\Longleftrightarrow{pH^{c,m}=-\log{\left(-C_{H}^{c,m}\right)}}
\end{equation}

\begin{equation}
f_{Hb}=1+b\,\frac{Q_{Hb}}{V_w }+{c}\left(\frac{Q_{Hb}}{V_w }\right)^2
\end{equation}

Pourcentage d'hémoglobine dans le sang (dont on a besoin pour déterminer les $dCmdt$)

\begin{equation}
Ht = Ht_0\,\exp{(V_w-V_w^0)}
\end{equation}

Concentration extra-cellulaire de H (dont on a besoin pour déterminer $FluxHA$)

\begin{equation}
C_{H}^{m}={K_B}\times{\frac{C_{HB}^{m}}{C_{B}^{m}-C_{HB}^{m}}}
\end{equation}

\subsection{Equations liées aux flux}
\subsubsection*{Équations des différents flux}

\begin{equation}
\Phi_{Na}^{P}={-\Phi_{max}^{Na}}\times{\left(\frac{Q_{Na}/V_w}{Q_{Na}/V_w+0.1\left(1+\frac{Q_{K}/V_w}{8.3}\right)}\right)^3}\times{\left(\frac{C_{K}^{m}}{C_{K}^{m}+0.1\left(1+\frac{C_{Na}^{m}}{18}\right)}\right)^2}
\end{equation}

\begin{equation}
\Phi_{K}^{P}=-\frac{\Phi_{Na}^{P}}{1.5}
\end{equation}

\begin{equation}
\Phi_{Na}^{L}={-P_{Na}^{L}}\times{\left(\frac{Q_{Na}}{V_w}-C_{Na}^{m}\right)}
\end{equation}

\begin{equation}
\Phi_{K}^{L}={-P_{K}^{L}}\times{\left(\frac{Q_{K}}{V_w}-C_{K}^{m}\right)}
\end{equation}

\subsubsection*{Équations des flux électro-diffusifs}

\begin{equation}
\Phi_{Na}^{G}={-P_{Na}^{G}}\times{\frac{FE}{RT}}\times{\left(\frac{Q_{Na}/V_w-{C_{Na}^{m}}\times{\exp{\frac{-FE}{RT}}}}{1 - \exp{\frac{-FE}{RT}}}\right)}
\end{equation}

\begin{equation}
\Phi_{K}^{G}={-P_{K}^{G}}\times{\frac{FE}{RT}}\times{\left(\frac{Q_{K}/V_w-{C_{K}^{m}}\times{\exp{\frac{-FE}{RT}}}}{1 - \exp{\frac{-FE}{RT}}}\right)}
\end{equation}

\begin{equation}
\Phi_{A}^{G}={+P_{A}^{G}}\times{\frac{FE}{RT}}\times{\left(\frac{Q_{A}/V_w-{C_{K}^{m}}\times{\exp{\frac{+FE}{RT}}}}{1 - \exp{\frac{+FE}{RT}}}\right)}
\end{equation}

\subsubsection*{Équations du cotransporteur Na:K:A}

\begin{equation}
\Phi_{A}^{Co}={-k_{Co}}\times{\left(\frac{{Q_{A}^2}\times{Q_{Na}}\times{Q_{K}}}{V_w^4}-{d}\times{\left(C_{A}^{m}\right)^2}\times{C_{Na}^{m}}\times{C_{K}^{m}}\right)}
\end{equation}

\begin{equation}
\Phi_{Na}^{Co}=\Phi_{K}^{Co}=\frac{\Phi_{A}^{Co}}{2}
\end{equation}

\subsubsection*{Équations du flux HA}

\begin{equation}
\Phi_{H}^{HA}=\Phi_{A}^{HA}={-k_{HA}}\times{( \frac{{Q_A}\times{Q_H}}{V_w^2} - {C_{A}^{m}}\times{C_{H}^{m}}})
\end{equation}\\

\subsection{Equations du système d'équations différentielles}

\begin{empheq}[left=\empheqlbrace]{align}
&\frac{dQ_{Na}}{dt}=\Phi_{Na}=\Phi_{Na}^{P}+\Phi_{Na}^{L}+\Phi_{Na}^{G}+\Phi_{Na}^{Co}\label{eq-1}\\
&\frac{dQ_K}{dt}=\Phi_{K}=\Phi_{K}^{P}+\Phi_{K}^{L}+\Phi_{K}^{G}+\Phi_{K}^{Co}\\
&\frac{dQ_A}{dt}=\Phi_{A}=\Phi_{A}^{G}+\Phi_{A}^{HA}+\Phi_{A}^{Co}\\
&\frac{dQ_H}{dt}=\Phi_{H}=\Phi_{H}^{HA}\\
&\frac{dV_w}{dt}= \frac{\dfrac{dQ_Na}{dt}+\dfrac{dQ_K}{dt}+\dfrac{dQ_A}{dt}}{C_{Na}^{m}+C_{K}^{m}+C_{A}^{m}+C_{B}^{m}+C_{Y}^{m}}\\
&\frac{dC_{Na}^m}{dt}={\frac{Ht}{1 - Ht}}\times{\left(C_{Na}^m-\frac{dQ_{Na}}{dt}\right)}\\
&\frac{dC_{K}^m}{dt}={\frac{Ht}{1 - Ht}}\times{\left(C_{K}^m-\frac{dQ_{K}}{dt}\right)}\\
&\frac{dC_{A}^m}{dt}={\frac{Ht}{1 - Ht}}\times{\left(C_{A}^m-\frac{dQ_{A}}{dt}\right)}\\
&\frac{dC_{HB}^m}{dt}={\frac{Ht}{1 - Ht}}\times{\left(C_{HB}^m-\frac{dQ_{H}}{dt}\right)}\\
&\frac{dC_{B}^m}{dt}={\frac{Ht}{1 - Ht}}\times{C_{B}^m}\\
&\frac{dC_{Y}^m}{dt}={\frac{Ht}{1 - Ht}}\times{C_{Y}^m}\\
\end{empheq}



\section{Implémentation}   






\section{Application à des cas concrets}     

\subsection{Cas 1} 

\subsection{Cas 2} 

\section{Les limites du modèle}


\section{Conclusion} 

+ ( bibliographie ? ou juste citer l'article dans l'intro)
+ ( lien du code ? Si posté sur internet )
 

% \paragraph{Titre}           % Toutes petites sections (le nom \paragraph
                              % n'est pas très bien choisi)

% \subparagraph{Titre}        % La dernière

% \appendix                   % Commençons les annexes

% \section{Titre}             % Annexe A

% \section{Titre}             % Annexe B

% \listoffigures              % Table des figures

% \listoftables               % Liste des tableaux

\end{document}
