% Tout ce qui est mis derrière un « % » n'est pas vu par LaTeX
% On appelle cela des « commentaires ».  Les commentaires permettent de
% commenter son document - comme ce que je suis en train de faire
% actuellement - et de cacher du code - cf. la ligne \pagestyle.

\documentclass[a4paper]{article}

% Options possibles : 10pt, 11pt, 12pt (taille de la fonte)
%                     oneside, twoside (recto simple, recto-verso)
%                     draft, final (stade de développement)

\usepackage[utf8]{inputenc}   % LaTeX, comprends les accents !
\usepackage[T1]{fontenc}      % Police contenant les caractères français
\usepackage[english,french]{babel}  % Placez ici une liste de langues, la dernière étant la langue principale

\usepackage[a4paper]{geometry}% Réduire les marges
%\pagestyle{headings}        % Pour mettre des entêtes avec les titres des sections en haut de page

\usepackage{amsmath} % package pour écrire des maths

\title{Modèle du globule rouge}           % Les paramètres du titre : titre, auteur, date
\author{Meyer Léo \and Klay Léna}
\date{}                       % La date n'est pas requise (la date du jour de compilation est utilisée en son absence)

\sloppy                       % Ne pas faire déborder les lignes dans la marge




\begin{document}

\maketitle                    % Faire un titre utilisant les données de  \title, \author et \date


\subsection*{Introduction}

But du programme, questions biologiques…

Petit schéma du globule rouge + échanges ioniques


\tableofcontents              % Table des matières

% \part{Titre}                % Commencer une partie...

\section{Notation}               % Commencer une section, etc.

Il existe des constantes qui bougent et d'autres qui ne bougent pas, des variables....

\subsection{Variables}         % Section plus petite


% \subsubsection{Titre}       % Encore plus petite

\subsection{Constantes qui ne varient jamais}

\begin{tabular}{lcrlcr}

\textbf{Constante} & \textbf{Abréviation} & \textbf{Valeur} & \textbf{Unité} \\
Taux d'hématocrite & Ht & 0.1 \\
Volume intracellulaire initial & Vw0 & 0.7  \\  
Quantité intracellulaire d'hémoglobine &  QHb & 5 & mmol/l \\

\end{tabular}

\subsection{Constantes qui varient selon les cas}

\section{Écriture du système d'équations différentielles}   


\subsection{Équations différentielles}

\subsection{Autres équations du modèle}

%E = - R*T/F * np.log(( PGNa*QNa/Vw + PGK*QK/Vw + PGA*CmA ) / ( PGNa*CmNa + PGK*CmK + PGA*QA/Vw ))

\begin{equation}
E = \frac{RT}{F}\log{(\frac{PGNa*QNa/Vw + PGK*QK/Vw + PGA*CmA}{PGNa*CmNa + PGK*CmK + PGA*QA/Vw})}
\end{equation}


\section{Application à des cas concrets}     

\subsection{Cas 1} 

\subsection{Cas 2} 

\section{Les limites du modèle}


\section{Conclusion} 

+ ( bibliographie ? ou juste citer l'article dans l'intro)
+ ( lien du code ? Si posté sur internet )
 

% \paragraph{Titre}           % Toutes petites sections (le nom \paragraph
                              % n'est pas très bien choisi)

% \subparagraph{Titre}        % La dernière

% \appendix                   % Commençons les annexes

% \section{Titre}             % Annexe A

% \section{Titre}             % Annexe B

% \listoffigures              % Table des figures

% \listoftables               % Liste des tableaux

\end{document}

